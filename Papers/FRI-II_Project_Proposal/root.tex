%%%%%%%%%%%%%%%%%%%%%%%%%%%%%%%%%%%%%%%%%%%%%%%%%%%%%%%%%%%%%%%%%%%%%%%%%%%%%%%%
%2345678901234567890123456789012345678901234567890123456789012345678901234567890
%        1         2         3         4         5         6         7         8

\documentclass[letterpaper, 10 pt, conference]{ieeeconf}  % Comment this line out if you need a4paper

%\documentclass[a4paper, 10pt, conference]{ieeeconf}      % Use this line for a4 paper

\IEEEoverridecommandlockouts                              % This command is only needed if 
                                                          % you want to use the \thanks command

\overrideIEEEmargins                                      % Needed to meet printer requirements.

%In case you encounter the following error:
%Error 1010 The PDF file may be corrupt (unable to open PDF file) OR
%Error 1000 An error occurred while parsing a contents stream. Unable to analyze the PDF file.
%This is a known problem with pdfLaTeX conversion filter. The file cannot be opened with acrobat reader
%Please use one of the alternatives below to circumvent this error by uncommenting one or the other
%\pdfobjcompresslevel=0
%\pdfminorversion=4

% See the \addtolength command later in the file to balance the column lengths
% on the last page of the document

% The following packages can be found on http:\\www.ctan.org
%\usepackage{graphics} % for pdf, bitmapped graphics files
%\usepackage{epsfig} % for postscript graphics files
%\usepackage{mathptmx} % assumes new font selection scheme installed
%\usepackage{times} % assumes new font selection scheme installed
%\usepackage{amsmath} % assumes amsmath package installed
%\usepackage{amssymb}  % assumes amsmath package installed
\usepackage{graphicx}
\graphicspath{ {./images/} }
\usepackage{subcaption}
\usepackage{url}
\usepackage{booktabs}
\usepackage{hyperref}

\title{\LARGE \bf
Adding LLM-based Complex Query Support to Semantically Labelled Maps}


\author{Raghav Rangan, Siddh Bamb, Kevin Zhao, Jerry He% <-this % stops a space
}
%\thanks{*This work was supported by...}% <-this % stops a space
%\thanks{$^{1}$Christina Petlowany is with the Cockrell School of Engineering,
%        The University of Texas at Austin, Austin, TX 78712, USA
%        {\tt\small cpetlowany@utexas.edu}}%
%\thanks{$^{2}$Justin Hart is with the College of Natural Sciences, The %University of Texas at Austin,
%        Austin, TX 78712, USA
%        {\tt\small justinhart@utexas.edu}}%
%}

% todo other authors


\begin{document}



\maketitle
\thispagestyle{empty}
\pagestyle{empty}


%%%%%%%%%%%%%%%%%%%%%%%%%%%%%%%%%%%%%%%%%%%%%%%%%%%%%%%%%%%%%%%%%%%%%%%%%%%%%%%%
\begin{abstract}
    The purpose of our project is to design and implement an LLM-based system to take a query regarding the internal state of a mapped space and answer it using labeled map data. Our goal is to make a robotic system capable of relating objects to each other both semantically and positionally in order to make it possible for it to answer questions or simply describe with relevant detail the space it has mapped. Our end goal is a system that is capable of describing a space in a way a human would.
\end{abstract}
    
\section{Introduction}
    The basis for our idea was the Robot Tour Guide project that was proposed to us by Dr. Hart. however, we felt like the use of the LLM could be expanded from simply interacting with a person, and could interface with the robot's mapping software to support more effective, contextual interaction. To achieve this, we first need to take a look at how the lab's current semantic mapping software works and outputs information. We will then look into CLIP, and how it can be used to draw relations between objects while labeling them from a segmented input. Lastly, we will be looking at joint embeddings capable of encoding aspects of an image (in our case, the output of the map) into a set of vectors, and decoding this set into a meaningful sentence, or set of sentences regarding a room. a more traditional LLM API, like Llama or GPT can be layered over this to support the frontend interaction. Our goal is for the frontend interaction to be as similar to a human's description of the space as possible.

\section{Background}
    As part of our preliminary research, we looked into how CLIP, CLIPSeg, and joint embedding spaces work. CLIP is a contrastive learning image model that takes advantage of a joint embedding space mapped onto by both a text encoder and image encoder. By training the encoders to map semantically similar text and images to similar values and push away dissimilar ones, the model can then pick what text matches the image most accurately. This allows for impressive zero shot performance, much better than previous models like ImageNet. CLIPSeg adds an additional decoder layer on top of this, taking the CLIP output and decoding it to segment out the desired part of the image.
    \\
    NEED TO ADD CITATIONS HERE AND METHODOLOGY



\section{Methodology}
    We will have three layers of the entire implementation: Semantic map - joint embedding - LLM API. This pipeline will support the entire interaction process. The majority of our work will deal with the creation of the joint embedding, and figuring out how to make it describe the map in enough detail so the LLM layer has a well-sized body of information to answer questions from. To help us in this process, we will look into how we can use CLIP/CLIPSeg to derive meaningful semantic relations, and potentially build off of CLIP's joint embedding space for text and image. This is applicable to our project since we can take a picture of the room, run it through CLIP/CLIPSeg, and derive useful semantic information that can be fed to the LLM, although additional training/modification of the joint mapping spaces may be necessary.


\section{Experimental Setup}
    Since our goal is to make our system's interaction feel as human as possible, our experiment will involve two types of tests. The first will be evaluation metrics of the system itself, through inbuilt network evaluation mechanisms. The second will be a practical test which can be done once the system is operational. We will have subjects interact with a human describing the space as well as our frontend, without knowing which one they are talking to. We will then ask them which they thought was which. Statistical tests can then be run on this data to see if there was a statistically signficant result (people were able to recognize which interaction was which correctly). A significant difference would mean our system is distinguishable from a human's description, which is a negative. We are looking for an insignificant result from this test, which would mean our system's output is highly similar to a real person's description. 


\section{Conclusion}
Conslusion

% uncomment this >>> when we actually cite stuff
%\bibliographystyle{IEEEtran}
%\bibliography{bibliography}


\end{document}

% todo read paper Blake holman Watch where you're going
